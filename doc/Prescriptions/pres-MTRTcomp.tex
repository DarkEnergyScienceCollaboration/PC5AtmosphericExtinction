%%%%%%%%%%%%%%%%%%%%%%%%%%%%%%%%%%%%%%%%%
% Beamer Presentation
% LaTeX Template
% Version 1.0 (10/11/12)
%
% This template has been downloaded from:
% http://www.LaTeXTemplates.com
%
% License:
% CC BY-NC-SA 3.0 (http://creativecommons.org/licenses/by-nc-sa/3.0/)
%
%%%%%%%%%%%%%%%%%%%%%%%%%%%%%%%%%%%%%%%%%

%----------------------------------------------------------------------------------------
%	PACKAGES AND THEMES
%----------------------------------------------------------------------------------------

\documentclass{beamer}

\usepackage{hyperref}

\mode<presentation> {

% The Beamer class comes with a number of default slide themes
% which change the colors and layouts of slides. Below this is a list
% of all the themes, uncomment each in turn to see what they look like.

%\usetheme{default}
%\usetheme{AnnArbor}
%\usetheme{Antibes}
%\usetheme{Bergen}
%\usetheme{Berkeley}
%\usetheme{Berlin}
%\usetheme{Boadilla}
%\usetheme{CambridgeUS}
%\usetheme{Copenhagen}
%\usetheme{Darmstadt}
%\usetheme{Dresden}
%\usetheme{Frankfurt}
%\usetheme{Goettingen}
%\usetheme{Hannover}
%\usetheme{Ilmenau}
%\usetheme{JuanLesPins}
%\usetheme{Luebeck}
\usetheme{Madrid}
%\usetheme{Malmoe}
%\usetheme{Marburg}
%\usetheme{Montpellier}
%\usetheme{PaloAlto}
%\usetheme{Pittsburgh}
%\usetheme{Rochester}
%\usetheme{Singapore}
%\usetheme{Szeged}
%\usetheme{Warsaw}

% As well as themes, the Beamer class has a number of color themes
% for any slide theme. Uncomment each of these in turn to see how it
% changes the colors of your current slide theme.

%\usecolortheme{albatross}
%\usecolortheme{beaver}
%\usecolortheme{beetle}
%\usecolortheme{crane}
%\usecolortheme{dolphin}
%\usecolortheme{dove}
%\usecolortheme{fly}
%\usecolortheme{lily}
%\usecolortheme{orchid}
%\usecolortheme{rose}
%\usecolortheme{seagull}
%\usecolortheme{seahorse}
%\usecolortheme{whale}
%\usecolortheme{wolverine}

%\setbeamertemplate{footline} % To remove the footer line in all slides uncomment this line
%\setbeamertemplate{footline}[page number] % To replace the footer line in all slides with a simple slide count uncomment this line

%\setbeamertemplate{navigation symbols}{} % To remove the navigation symbols from the bottom of all slides uncomment this line
}

\usepackage{graphicx} % Allows including images
\usepackage{booktabs} % Allows the use of \toprule, \midrule and \bottomrule in tables

%----------------------------------------------------------------------------------------
%	TITLE PAGE
%----------------------------------------------------------------------------------------

\title[MT-RT Comparison]{Prescriptions to compare Modtran and LibRadTran} % The short title appears at the bottom of every slide, the full title is only on the title page

\author{
Dagoret-Campagne, Sylvie  ,  LAL, % Your name
\textit{dagoret@lal.in2p3.fr} % Your email address 
\\
\and
Gilmore, David Kirk , SLAC,
\textit{dkg@slac.stanford.edu} % Your email address
\and
\\
For PCWG team of DESC-LSST collaboration 
}

\date{\today} % Date, can be changed to a custom date

\begin{document}

\begin{frame}
\titlepage % Print the title page as the first slide
\end{frame}

\begin{frame}
\frametitle{Overview} % Table of contents slide, comment this block out to remove it
\tableofcontents % Throughout your presentation, if you choose to use \section{} and \subsection{} commands, these will automatically be printed on this slide as an overview of your presentation
\end{frame}

%----------------------------------------------------------------------------------------
%	PRESENTATION SLIDES
%----------------------------------------------------------------------------------------


%%%%%%%%%%%%%
\section{Introduction}
%%%%%%%%%%%%%

%---------------------------------------------------------------------
% Slide 1
%---------------------------------------------------------------------
\begin{frame}
\frametitle{Motivation}
\begin{itemize}
\item Want to use precise atmospheric models to correct LSST images from atmospheric attenuation,
\item Target accuracy : 1\% on flux or 10 mmag on Magnitude imply a 1\% accuracy on air-transparency,
\item From a few precise auxiliary measurements, can we infer the air transparency over the LSST full wavelength range ?
\item It is difficult to compare simulation codes with well calibrated data (including with known air  transparency).
\item So the first stage of checking the simulation validity is to compare the prediction of Modtran-LibRadTran
\end{itemize}
\end{frame}


%---------------------------------------------------------------------
% Slide 2
%---------------------------------------------------------------------
\begin{frame}
\frametitle{Evaluation of air transparency simulation codes}
\begin{itemize}
\item Modtran : Very popular code in the US, natural legacy, the reference. 
\item LibRadTran: More recent implementation, easy to use, open code, need to be evaluated wrt Modtran.
\end{itemize}
\end{frame}


%---------------------------------------------------------------------
% Slide 3
%---------------------------------------------------------------------
\begin{frame}
\frametitle{But Compare exactly what ?}
We should first share the same input assumption for comparing the output results.
\begin{itemize}
\item What kind of "thermodynamical" atmosphere (gas) should we use as input  ?  $\rightarrow$ define the \textbf{atmospheric model},
\begin{itemize}
\item AltitudeTemperature profile, 
\item AltitudePressure profiles,
\item Molecular species profiles (at least those that matter)
\end{itemize}
\item Similar light-molecules interaction models: $\rightarrow$ define the \textbf{absorption models},
\item Check the air transparency for the same grid in $(\lambda,z)$ points.
\end{itemize}
To agree with the parameters, we have to define the common parameters choice among hundreds of parameters to be tuned $\rightarrow$ define \textbf{prescriptions}.
\end{frame}

%%%%%%%%%%%%%%%%%%%%%%%%%%%%%%%%%%%%%%%%%%%%%%%%%%%%%%%%%%%%%%%%%%%%%%

%%%%%%%%%%%%%%%
\section{The prescription}
%%%%%%%%%%%%%%%%%

%---------------------------------------------------------------------
% Slide 4
%---------------------------------------------------------------------
\begin{frame}
\frametitle{Data points of the grid}
\begin{block}{Altitude}
$h=2.75$~km.
\end{block}

\begin{block}{Wavelength range}
$\lambda \in [250,1200]$~nm, $\Delta \lambda=$1~nm
\end{block}

\begin{block}{Airmass range}
$z$ computed for 31 points in steps of 0.1~: $z \in [1,3]$
\end{block}

\end{frame}

%---------------------------------------------------------------------
% Slide 5
%---------------------------------------------------------------------

\begin{frame}
\frametitle{The Atmospheric Profile}
Compare predictions for 2 realistic atmospheric models.
\begin{itemize}
\item US standard atmosphere $\rightarrow$ probably good for LSST, but perhaps  a little too wet
\item Subarctic winter $\rightarrow$ very dry air, probably well suited for LSST site
\end{itemize}

\textcolor{gray}{The above atmospheric models are both available in Modtran(Data Card : Card1.MODEL )  and LibRadTran (Data Card : atmosphere\_file.)}

\end{frame}


%---------------------------------------------------------------------
% Slide 6
%---------------------------------------------------------------------

\begin{frame}
\frametitle{The absorption models  in LibRadTran}
\begin{itemize}
\item {\bf Spectrally resolved  calculation},
\item {\bf Line-by-line calculation},
\item {\bf The correlated-k method},(Kato and Kato2),
\item {\bf The correlated-k method},(Fu and Liou),
\item {\bf Representative wavelengths parameterization (REPTRAN)},
\item {\bf Pseudo-spectral calculation adapted from LOWTRAN}. 
\item {\bf CRS: switching off spectral parameterization}. 
\end{itemize} 
\begin{block} {\bf Absorption models in LibRadTran~:} 
1) REPTRAN, 2) LOWTRAN,  3) CRS, 4) Fu 5) KATO,  6) KATO2,  7) KATO2.96 
\end{block}

\textcolor{gray}{The Data card for specifying the absorption model in LibRadTran is  \textit{mol\_abs\_param} . One should provides the values \textit{reptran fine} or  \textit{reptran medium}, \textit{reptran coarse} or \textit{lowtran},...}

\end{frame}


%---------------------------------------------------------------------
% Slide 6
%---------------------------------------------------------------------

\begin{frame}
\frametitle{The absorption models  in Modtran}

\begin{itemize}
\item {\bf Band models methods}, The Modtran or Lowtran models,
\item {\bf The correlated-k method}, The Modtran and Lowtran either in slow or medium modes.
\end{itemize} 

\begin{block} {\bf Absorption models in LibRadTran~:} 
1) MODTRAN band Model, 2) LOWTRAN Band Model ,  3) MODTRAN correlated-k  and slow speed,  4) MODTRAN correlated-k  and slow speed, 5) MODTRAN correlated-k  and medium speed
\end{block}

\textcolor{gray}{The Data card for specifying the absorption model in Modtran is  \textit{CARD1 MODTRN } . One should provides the values \textit{T,M,C,K,F,L} and \textit{CARD1 SPEED} is \textit{S} or blank. }
\end{frame}



%---------------------------------------------------------------------
% Slide 7
%---------------------------------------------------------------------

\begin{frame}
\frametitle{Comparison of light-atmospheric matter interaction processes}
\begin{itemize}
\item Simulate with only molecular scattering (Rayleigh scattering). 
\item Simulate pure molecular absorption.
\begin{block}{Variation of Precipitable Pressure Water $PWV$}
 absorption profiles  for 31 points $pwv$ $\in \left[0~{\rm mm},15~{\rm mm} \right]$ in steps of 0.5~mm\end{block}
 \begin{block}{Variation of Ozone $O_3$}
 absorption profiles for  21 points $O_3$ $\in \left[200~{\rm Dobson},600~{\rm Dobson} \right]$.
\end{block}
\item Simulate the combination of molecular scattering and molecular absorption. 
\begin{itemize}
\item perform similar variations in $PWV$ and $O_3$
\end{itemize}
\end{itemize} 
\end{frame}

%----------------------------------------------------------------------------------------------------------------------------------------------

%%%%%%%%%%%%%%%
\section{Practical organization}
%%%%%%%%%%%%%%%

%---------------------------------------------------------------------
% Slide 8
%---------------------------------------------------------------------
\begin{frame}
\frametitle{File naming : Part 1 }
\begin{block}{filename}
$P\_O\_\{rte\}\_\{atm\}\_\{proc\}\_\{mod\}\_zXX\_wvXX\_ozXX.extension $ 
\end{block}
{\tiny
\begin{itemize}
\item [{\bf P :} ] {\bf RT} or {\bf MT} for LibRadtran or Modtran,
\item [{\bf O :} ]  Observatory site {\bf LS} or {\bf HP}  or {\bf GM}  or {\bf MK} for LSST, OHP, Gemini South, Mauna Kea,...
\item [{\bf \{rte\} :}] {\bf pp} or  {\bf ps},
\item [{\bf \{atm\} :}]  {\bf us} or {\bf sw},
\item [{\bf \{proc\} :}] {\bf sc} for pure molecular scattering, {\bf ab} for pure molecular absorption, {\bf sa} for the combination of molecular scattering and absorption,
\item [{\bf \{mod\} :}] in LibRadTran :  {\bf rt} for Reptran model, {\bf lt} for Lowtran model, {\bf cr} for CRS model,  {\bf fu} for the Fu and Liou model, {\bf k2} for Kato2 model, and {\bf kt} for Kato model.
\item [{\bf \{mod\} :}] in ModTran :  {\bf mt} for Modtran band model), {\bf mk} for Modtran correlated-k model, {\bf lt} for Lowtran model.
\item [{\bf zXX :} ] Airmass $z$, where XX is the value of the airmass on 2 digit $XX=2\times z$,
\item [{\bf wvXX :} ] Precipitable water vapour $pwv$, where XX is the value of the $pwv$ on 3 digit $XXX=10\times pwv$, $pwv$ in mm unit,
\item [{\bf ozXX :} ] Ozone $oz$, where XX is the value of the $oz$ on 2 digit $XX=oz/10$, $oz$ is Dobson unit.
\end{itemize}

For example a filename for LibRadtran, for LSST could be~:
\begin{equation}
RT\_LS\_pp\_us\_sa\_rt\_z15\_wv030\_oz30.txt  \nonumber
\end{equation}
where $pwv=3$~mm and $oz=300$~Dobson unit and $z=1.5$.

}
\end{frame}



%---------------------------------------------------------------------
% Slide 9
%---------------------------------------------------------------------
\begin{frame}
\frametitle{File naming  : Part 2 }
Naming absorption models for Modtran
\begin{block}{filename}
$P\_O\_\{rte\}\_\{atm\}\_\{proc\}\_\{mod\}\_zXX\_wvXX\_ozXX.extension $ 
\end{block}
{\tiny
\begin{itemize}
\item [{\bf \{mod\} :}] 
{\tiny
\begin{itemize}
{\scriptsize
 \item {\bf mt} for Modtran band model (\textcolor{gray}{CARD1.MODTRN='T'}), 
 \item {\bf mm} for Modtran band model (\textcolor{gray}{CARD1.MODTRN='M'}),, 
 \item {\bf lt} for Lowtran band model (\textcolor{gray}{CARD1.MODTRN='L' or 'F'}), 
 \item {\bf mks} for Modtran correlate-k in slow mode (speed), (\textcolor{gray}{CARD1.MODTRN='K' or 'C and CARD1.SEED='S'}), 
 \item {\bf mkm} for Modtran correlate-k in medium mode (speed), (\textcolor{gray}{CARD1.MODTRN='K' or 'C and CARD1.SEED='M'}),
 }
 \end{itemize}
 }
\end{itemize}
For example a filename for Modtran, for LSST could be~:
\begin{equation}
MT\_LS\_pp\_us\_sa\_mt\_z15\_wv030\_oz30.txt  \nonumber
\end{equation}
where $pwv=3$~mm and $oz=300$~Dobson unit.
}
\end{frame}

%---------------------------------------------------------------------
% Slide 10
%---------------------------------------------------------------------


\begin{frame}
\frametitle{Hierarchy of directories}
\begin{table}
{\tiny
\begin{tabular}{ l l}
  rootdir &  top directory \\
 rootdir/RT/VXX/LS/pp/us/sc/ & pure molecular scattering \\
 rootdir/RT/VXX/LS/pp/us/ab/rt/wv/ & pure molecular absorption for varying $pwv$ \\
 rootdir/RT/VXX/LS/pp/us/ab/rt/oz/ & pure molecular absorption for varying $oz$ \\
 rootdir/RT/VXX/LS/pp/us/sa/rt/wv/ & molecular absorption and scattering for varying $pwv$ \\
 rootdir/RT/VXX/LS/pp/us/sa/rt/oz/ & molecular absorption scattering for varying $oz$ \\
 rootdir/RT/VXX/LS/pp/us/sc/ae/ & pure molecular scattering and default aerosols scattering \\
 rootdir/RT/VXX/LS/pp/us/sa/rt/ae/ & molecular absorption scattering and default aerosols scattering \\
 rootdir/RT/VXX/LS/pp/us/sc/as/ & pure molecular scattering and special aerosols scattering \\
 rootdir/RT/VXX/LS/pp/us/sa/rt/as/ & molecular absorption scattering and special aerosols scattering \\
\end{tabular}
}
\end{table}
$VXX$ is the version number of Radtrran or Modtran.
\end{frame}


%---------------------------------------------------------------------
% Slide 11
%---------------------------------------------------------------------

\begin{frame}
\frametitle{Status of the work}
{\tiny
\begin{itemize}
\item GitHub Repository created by Nicolas at \\
 \href{https://github.com/DarkEnergyScienceCollaboration/PC5AtmosphericExtinction}{https://github.com/DarkEnergyScienceCollaboration/PC5AtmosphericExtinction}
\item Prescription note written to proceed with air transparency simulations and posted at \\
 \href{https://github.com/DarkEnergyScienceCollaboration/PC5AtmosphericExtinction/tree/master/doc/Prescriptions}{https://github.com/DarkEnergyScienceCollaboration/PC5AtmosphericExtinction/tree/master/doc/Prescriptions}
\item Simulation with LibRadTran {\bf done} and at posted at (see README) \\
\href{https://github.com/DarkEnergyScienceCollaboration/PC5AtmosphericExtinction/tree/master/LibRadTran/simulations}{https://github.com/DarkEnergyScienceCollaboration/PC5AtmosphericExtinction/tree/master/LibRadTran/simulations}
\item Simulation with Modtran {\bf soon performed} and at posted at (see README) \\
\href{https://github.com/DarkEnergyScienceCollaboration/PC5AtmosphericExtinction/tree/master/ModTran}{https://github.com/DarkEnergyScienceCollaboration/PC5AtmosphericExtinction/tree/master/ModTran}
\end{itemize}
}

\begin{itemize}
\item Ready for Atmospheric properties Analysis,
\item Almost ready for comparison, at least comparing the different models inside LibRadTran.
\end{itemize}

\end{frame}





\begin{frame}
\Huge{\centerline{The End}}

\end{frame}

%----------------------------------------------------------------------------------------

\end{document} 